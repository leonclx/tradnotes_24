\subsection{Crevasse rescue}
\begin{itemize}
\item make a T, with sling just as deep as other
\item ensure dead man orthogonal to where rope runs, but also orthogonal to snow slope
\item first priority is sitting/holding securely. Then build deadman nearby
\item microtrax at top to attach to dead man
\item how good is snow? try putting fingers/fist into. denser is better
\item at crevasse, put sth under rope to prevent cutting deeper
\item unconscious victim? call emergency, reach victim quickly
\item if possible, drop biener w/ 2 strings, not always easy
\item put own lifeline prusik on tight rope
\item 3 to 1 if alone probably not enough, probably 5 to 1 needed
\item 7 to 1 careful not strangulating victim
\item practice w/ snow covered crevasse much different from bare crevasse
\item keep rope tight while walking
\end{itemize}

\subsection{Misc}
\begin{itemize}
\item Going up steep snow field: Put weight on top of pickel, don't grab shaft but lean on top. Most weight on feet
\item Edge of glacier is most dangerous, this is where most of the movement is. Keep away
\item Don't pull on big rocks near glacier edge
\end{itemize}
