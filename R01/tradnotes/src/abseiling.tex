\subsection{Walking down}
Walking down is usually preferred over abseiling. It's not always better, just typically faster and safer.
\begin{itemize}
\item Keep an eye out for cairns, but don't blindly follow them. Similar for other people.
\item If you lost the path, go back and try to find it again. Don't keep going into mistakes, turn around
\item Look a bit ahead to see where you should end up
\item In couloir, either go one for one (takes long) or stick together. Beware of loose rocks
\item Consider consequences of slip/fall: to next ledge (broken ankle) or off cliff (dead)
\item Judge terrain and act accordingly: crampons, pickaxe, short simulclimb/belay, 10m abseil..
\end{itemize}


\subsection{Abseiling}
Abseiling is dangerous and how most climbers die. It's usually done after climbing, when tired.
\begin{itemize}
\item Abseil anchor
	\begin{itemize}
	\item Check sunfaded, still supple/not stiff, check behind prussik for cuts/damage
	\item joining 2 pietons: single overhand fed thru for both, another one for abseil metal point (equalized). More good info \href{https://www.alpinesavvy.com/blog/retreat-anchors-alpine-climbing}{here}
	\item no metal/maillon rapide: one way ticket. OK for you, but rope burns prussik when pulling
	\item abseil on single snapper? possible, but check gate away from rock, be careful
	\item if you see bad/old sling, just cut it away
	\end{itemize}
\item Going back up on rope
	\begin{itemize}
	\item Only works in slow mode, with ATC teeth facing downwards
	\item Start by putting long prussik above ATC, for stepping in
	\item Step up, clip ATC guide mode ring to belay loop w/ screwgate
	\item Put backup knot below abseil prussik, remove abseil prussik
	\item Step up, pull rope through ATC, move foot prussik up. Repeat
	\end{itemize}
\item Joining knot (double rope), "european death knot"
	\begin{itemize}
	\item No crosses in knot, has to be dressed correctly
	\item Has to be very tight. pull on 4 strands individually
	\item Needs sufficient tail of 30-50cm. Don't make it longer, risk of setting up rappel on tail (deadly)
	\item Knot should be below maillon rapide, less risk of getting stuck
	\end{itemize}
\item Stopper knots
	\begin{itemize}
	\item Single overhand knot: Nico thinks not enough. Can come undone, rolls
	\item Barrel knot: Bit bigger, safe enough. Small enough not to get stuck
	\item Bigger knots: More risk of rope getting stuck from below
	\item Risky but possible: no stopper knots. Has to be clearly communicated. If next relais isn't visible/clear, put them in. Guides dont tie stopper knots usually, unless they dont see anchor
	\item Common mistake: pull rope thru, it falls down..oh nice it's already hanging. But: no stopper knot
	\end{itemize}
\item Rope control
	\begin{itemize}
	\item First abseiler doesn't remove prussik until 2nd is at relais
	\item Can extend prussik w/ 120cm sling if rope is slightly too short
	\item Risk in overhang or with wind: loosing control of rope
	\item Granite has high risk of getting rope stuck
	\item Rope stuck from below: don't pull it tight like a nut, flick it
	\item Keep ropes apart at relais, risk of getting twisted against eachother
	\item 1st abseiler can take rope with, coiled up on side of harness. Start on stopper knot end, coil up over head, use extended sling to clip to harness. Cams/Nuts tend to get stuck in rope carried like that. Can't get ends stuck like this.
	\item If 1st abseiler always has to solve rope spaghetti, it's not efficient
	\item Make shorter abseils if terrain is flat (reduce risk of rope stuck)
	\end{itemize}
\item Procedure
	\begin{itemize}
	\item Always partner check both stopper knots, the joining knot, which side to pull
	\item Abseil setup has to be clean: No twists, no prussik getting into ATC, no hair into ATC..
	\item 1st has to find next relais, keep control of rope ends, prepare next abseil
	\item When 1st arrives at next relais, should clip extra screwgate to share for lifelines
	\item 2nd has to not feed abseil rope through sharp V (rope stuck), clip lifeline on pull side
	\item Common mistake: 2nd arrives, clips lifeline. Immediately forgets which to pull. Use separate sling?
	\item At next relais, 1st can already prepare next abseil by putting pull side thru maillon, removing stopper knot on other rope
	\end{itemize}
\item Rope stuck above on one end, right after loosing other end? Tie one end to relais, prussik up stuck rope but place cams below. Rope suddenly gets unstuck, will fall in cams
\item Safety over efficiency, if you rush too much, it will cause mistakes and just end up costing more time
\end{itemize}

\subsection{Thunderstorm}
\begin{itemize}
\item In case there's a thunderstorm, just leave everything and get down
\item Leave cams behind, it's not worth your life. Leave the rope too
\item Stay away from via ferrata/metal. Throw metal stuff away, especially if you hear buzzing

\end{itemize}
