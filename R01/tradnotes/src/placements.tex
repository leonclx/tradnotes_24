Making good placements is the most important part of trad climbing. 
\begin{itemize}
\item All lobes of the cam must be within their working range, with good contact to rock
	\begin{itemize}
	\item On a C4 cam, if one lobe has no contact the whole cam comes out
	\item Totem cams are bodyweight rated for 2 lobes, 4 lobes is better
	\item Ideal engagement is at edge tip of cam facing directly down
	\item Overcammed (cam too big for crack) is safe, but risk of not getting cam back
	\item Undercammed (cam too small for crack) is dangerous
	\item Check for good contact area (solid rock, not just barely touching a rock tip)
	\end{itemize}
\item Make a placement before you anticipate a climbing move
	\begin{itemize}
	\item Similar to sports climbing, don't clip in the middle of the crux
	\item Clip before/after a climbing move, in a good position (physically and mentally)
	\item After a hard move, can place bomber next cam and then remove previous placement (runout)
	\item In some cracks, can move cam along as you go (somewhat risky)
	\end{itemize}
\item Place the stem of the cam in the direction you expect a fall/load
\item Tap the rock with the bottom of your palm to check if it's loose
\item Don't take a cam and try to jam it in somewhere. Look calmly for a crack and take the correct one
\item Cracks that open up towards the climber aren't good for placements because the cam looses engagement when it moves a tiny bit out of the wall
\item Cracks that open up towards the wall also aren't good because the cam can walk into the crack and loose engagement
\item Cams are asymmetric: One side has 2 wide apart lobes, the other 2 close together
	\begin{itemize} 
	\item Try both orientations, sometimes one works better
	\item Horizontal cracks are preferred with placing the wide apart lobes at the bottom (more stable)
	\end{itemize}
\item Don't make bad placements, it's a bad habit. Try a different size/crack, or climb up/down
\item Nuts need good surface area on the sides and a tuck to set them in. Extend with more than a quickdraw or they come out
\item Cam walking can be reduced with extended quickdraws. Especially bigger ones walk
\item You can practice cam placements on a bolted route or on boulder/scrambling terrain
\end{itemize}

\subsection{Removing placements}
\begin{itemize}
\item Stuck cam: Try getting the lobes to move. Use force but in the right way
\item Nuts: Place nut tool from below and give it a tap in the direction where it comes out
\item Tricam: push in a bit and then use a nut tool to fish it out
\end{itemize}

\subsection{Trad anchor}
\begin{itemize}
\item Placements are rated from 1-4, higher being better
\item A trad anchor needs at least 8 points (summing the points of all placements)
\item Assume factor 2 lead fall and a bad anchor $\rightarrow$ both climbers fall all the way down. Anchor is important
\item 2 really good cams is perfectly fine, no need to make 4..6..8 placements
\item Any material used in the anchor is unavailable for the next leader, but better to place a bit more and then discuss with partner
\item Sling anchors need to be counterloaded w/ bodyweight to prevent being pulled up/away from rock
\item Can place extra cam to hold upwards force for leader fall. Pre-tensioning usually not needed.
\item \href{https://www.alpinesavvy.com/blog/try-a-girth-hitch-at-the-master-point}{Girth hitch master point}: Quick and easy to remove, but needs "magic X" (twisting one of the lines)
\end{itemize}


\subsection{Types of protection devices}
\begin{itemize}
\item SLCDs (spring loaded camming device), anything with a spring that returns to an expanded form automatically. Cams/Friends, ballnuts, DMM Dragons, Camalot C4/Z4, Totems..
\item passive protection (nuts, wires)
\item semi-passive devices (tricams)
\end{itemize}

