A good practice for tour planning is to get information from multiple sources. Don't trust one topo, but look it up on camptocamp. Check camptocamp outings, \textbf{ask the hut guardian}, check UK climbing forums. Topos at the hut can have additional handwritten information.
\subsection{Weather}
\begin{itemize}
\item Don't only look at the forecast, also look at how predictable the weather is
\item Meteoblue Arome (in multimodel) is best for predicting storms
\item Just rain is annoying but not that dangerous. Thunder is really dangerous
\item Use both meteoswiss/meteoschweiz and meteoblue
\item During the day, keep looking up. Are clouds moving towards? Dark/gray tall clouds (cumulonimbus) or high faint ones (cirrus)?
\item Better to check weather after breakfast instead of during/before. Can still change
\end{itemize}

\subsection{3x3}
Columns: Terrain, Conditions, Team. Rows: At home, At the start of the route, During the route.\\
See also \href{https://nkbv.nl/kenniscentrum/3-x-3-zomer-risicomanagement-in-alpien-terrein.html}{NKBV 3x3 zomer}

\subsection{Plan B}
It's useful to have a backup plan ready in case some unpredictable thing changes. Some examples:
\begin{itemize}
\item Weather prediction at 06:00 is better than last night vs worse/earlier rain
\item There's a slow team in front vs there's an empty route
\item Team member had a slight headache which remained vs team member feels a lot better
\end{itemize}

\subsection{Gear planning}
\begin{itemize}
\item Cams of size 4 and above are only needed if it says in topo
\item "1 rack" in topo usually assumes double middle sizes (gray, purple, green)
\item Better to bring too much cams and not need them than too little and be unsafe
\item Don't trust the topo too much regarding bolt location, gear quantity etc
\item Likewise, better to bring crampons/pickel and not need them than fall off snowfield
\item Leave C shoes/crampons below and abseil down? Or take with and walk down?
\end{itemize}

\subsection{Time estimation}
For time estimation it's good to leave margin because some things cannot be predicted precisely beforehand. For example rain could start at 12:00 or at 15:00, the 4c in the topo could feel like a 5c.\\
As the day goes on, those uncertainties usually become more clear.
\subsubsection{Approach/Descent}
\begin{itemize}
\item Times for approach and descent are usually listed in the topo
\item If the path is difficult to find, it's easy to loose 15min-30min (or more)
\item Naismiths hiking rule states 1h per 5km flat and 1h for 600m height (summed together)
\item Using only height meters and steep-ish terrain: 300-400m/h
\item Hiking with a heavy backpack is slower, especially when ascending
\item Some apps (eg Swisstopo) allow drawing lines to estimate hiking time
\item At the start/end of the route you can expect 5-10min to uncoil rope, put on shoes, organize harness
\item Well-prepared, aligned team: 30min for wakeup to out door. Inefficient: 1h, 1h30..
\item Taking a food break right before 100m height gain is miserable
\end{itemize}

\subsubsection{Climbing}
The general rule of thumb is 30min for climbing and 15min for abseiling each pitch.
\begin{itemize}
\item If the climbing is hard, more time is needed. Example 1st pitch 6a+ slab, wet shoes, 3 leadfalls: 1h
\item Easy pitches that have unclear direction can also take a bit longer (route finding). Rope drag!
\item Pitch length can vary; some are 45m, some are only 20-25m
\item Is it a straightforward sport style pitch or an alpine style arrête?
\item An efficient team with a clear line can climb a lot faster than 30min per pitch
\item If there's a slow group in front it can cost hours
\item Consider the climbing conditions: wet slab/crack, cold hands/feets, type of climbing
\item Abseiling can go 10min per pitch, but if the rope gets stuck once that can cost 15-30min
\item Can set an alarm on phone for the turnaround time
\end{itemize}



