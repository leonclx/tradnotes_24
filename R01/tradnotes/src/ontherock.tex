\subsection{Approach/Descent}
\begin{itemize}
\item Make clear decisions and align but don't stand around for minutes doing nothing
\item Move as a team. Say "stop that's the wrong way" instead of "I found a better way but you do you"
\item Take off some load (ropes/water) if a teammate is slower than the rest
\item Conditions affect approach: Snowfields? Wet rock?
\item Often conflicting information, different options for approach/descent
\item Transition between scrambling and climbing can be fluid, free soloing is dangerous
\item Walking efficiently
	\begin{itemize}
	\item Look a few steps ahead, not just at your feet
	\item Aim at the tops of rocks, so you can roll over
	\item Try to keep momentum, don't get stuck against uphill rock
	\item Don't overdo it, rocks might be loose
	\end{itemize}
\item Better to put on helmet too early than to forget it later on. Similar with harness
\end{itemize}

\subsection{During the climb}
\subsubsection{General tips}
\begin{itemize}
\item Belaying more fluidly: hold follower rope under tension against knee, can feel when need to take in
\item Keeping anchor clean is really important, don't put unnecessary bieners, take out ATC when done etc
\item If there's spaghetti at belayer soon after starting, stop and turn around
\item Traverse trick
	\begin{itemize}
	\item In traverse: bolt .. tricky move ....................... next bolt
	\item Put a prussik thru last bolt, other end clipped into rope
	\item Allows second climber to safely climb tricky move, then untie/remove prussik
	\item Prussik needs to go beyond crux (5m usually enough)
	\item Can also do this for yourself if leader didn't do it
	\item Easier and faster to place piece after tricky move, not always possible on slab
	\end{itemize}
\item When calling 112, tell them where and that high mountain rescue is needed
\item For microtraxion, have to be sure nothing goes into/against the teeth (ice/rock/dirt/stone)
\item Belaying with microtraxion is ok, but has to always be tight esp. near relais. See \href{https://www.petzl.com/US/en/Sport/Belaying-the-second-with-a-MICRO-TRAXION--beware-of-any-fall?ProductName=MICRO-TRAXION}{Petzl tech tip}
\item Snapper can't be used for anchor main point, but for 2nd point it's fine. Ensure gate not against rock
\item When aiding, clip lifeline to piece instead of asking for block from belayer. With a block, the piece gets loaded with twice the force compared to lifeline (both climber and belayer hang on it instead of just climber).
\item Evaluate anchor before loading it. Don't hang in something and then check if it's good..
\item There's a way to make an HMS (munter hitch) autoblock by putting a screwgate into it, see \href{https://www.alpinesavvy.com/blog/the-auto-locking-munter-hitch}{here}
\item In limestone it's harder to place than in granite. 6kN small cams not great for anchor
\item There's no place for ego on the mountain. Important to have discussion moments when needed
\item It's nice to take a hardshell as a leader, especially if the weather is unpredictable
\item Right above relais, need more pieces. Towards end of pitch, can have more runout
\item Consider consequences of lead fall: Falls into bergschrund or on ice? Pulls belayer down snowfield or into wall? Fall onto ledge or into overhang? How much runout is ok?
\item If there's a pieton, no reason to skip it. Use short quickdraws if leadfall risk
\item For a hard pitch, consider leaving weight at follower (waterbottle/food/backpack)
\item In a chimney, clip your backpack below your legs
\end{itemize}

\subsubsection{Harness organization}
Having a clearly organized harness makes climbing safer and saves time.
Imagine struggling in a difficult position, far above your last piece, desperately searching for the right cam on your messy harness...
\begin{itemize}
\item Place equipment you need in front of your harness, where it's easy to reach
\item Left/right is less important, keep some balance. For most, right side is easier to reach
\item Put less needed/bigger equipment towards the back where it gets less in the way
\item Anticipate equipment needed for pitch: finger crack cam size, many slings for arrête, big offwidth..
\item Put things you don't need in your bagpack. Example 5m prussik, big cams if not needed etc
\item Try to keep things clean/organized from the start instead of "doing it later", it's less effort overall
\item Gates out is objectively better than gates in because carabieners take space if the spine is at the hip
\item Empty your harness for the approach/descent because cams/slings can get stuck (falling risk)
\item Bringing too much equipment makes the harness messy. 5 screwgates are usually plenty, no need for 20 quickdraws etc
\end{itemize}


\subsubsection{Helping a weak follower}
\begin{itemize}
\item Tell them to prussik up. Useful for lots of friction in overhang or for hanging precisely near stuck cam
\item "The frog" aka "leg workout": put prussik on climber side and squat up. Good at ledges
\item Double rope: block on one end, climber pulls on other (needs communication)
\item 3-1 usually not enough, can go to 5-1 directly. Sometimes microtrax and own body as counterweight is enough
\item Quick 5-1
	\begin{itemize}
	\item Attach microtrax to tight climber rope with screwgate, 60cm sling basket hitched to anchor
	\item Put backup knot on brake side (5m, figure 8, clip to anchor)
	\item Wriggle ATC biener to unload into microtrax
	\item Rest of 5-1: line fixed at anchor, going thru biener of prussik on victim rope. Other end biener, fed thru belay side. See \href{https://www.alpinesavvy.com/blog/6-1-compound-pulleys-in-the-real-world}{here}
	\end{itemize}
\end{itemize}


\subsubsection{Efficiency/Faster}
\begin{itemize}
\item Safety first, speed should come naturally, but never at cost of safety
\item Just fast is not the proper word..more about controlled, efficient, focussed, systematic
\item Gain time not by climbing faster but by everything else..looking 3min at placements, taking long to build anchors, taking 15min to rack up before climbing
\item For taking decisions, not stand around for minutes doing nothing, also not just doing anything. Decisive but discussed
\item ATC efficiency trick: always have 2 screwgates on atc, clip anchor side to harness. Clip to anchor after stand, close screwgate. Do this before pulling rope in
\item Just do stuff. Don't be dreaming. Split tasks (one puts on climbing shoes, other feeds through rope)
\end{itemize} 

\newpage
\subsection{Lifeline considerations}
When arriving at an anchor, how do you attach yourself to it?
\subsubsection{Requirements}
\begin{itemize}
\item Adaptable in length. Can be 20cm for hanging belay or 2-3m for nice big ledge. Usually 20cm-1m
\item Dynamic, in case belayer slips while trying to move around a bit (fall on static sling very dangerous)
\item Quick and easy to use, not make harness too messy/get stuck
\item For abseiling/rescue/solving rope spaghetti: not be in rope system
\item Safe: 8mm dyneema w/ knots is weak and easily burned by rope running over it
\end{itemize} 

\subsubsection{Options}
\begin{itemize}
\item Extendable dynamic lifeline, for example petzl connect, kong slide. Always on belay loop
\item Static lifeline with multiple fixed lengths (120cm aramid sling, grivel belay chain, etc)
\item Using the rope with a clove hitch
\end{itemize} 


\subsubsection{Dynamic Lifeline}
\begin{itemize}
\item Nice when you get to anchor. Clip bolt, pull on rope, you can hang
\item Adaptable, but not as flexible as rope. Good for aiding
\item Dynamic, safe against slip/fall at anchor
\item Heavy, takes up space on harness. Sometimes annoying to put away
\item Outside of rope system. Safe against burning through
\end{itemize}

\subsubsection{Static sling lifeline}
\begin{itemize}
\item Nice when you get to anchor, bit worse than dynamic lifeline but still easy
\item 2-3 fixed lengths, belay can get uncomfortable
\item Not safe against slip/fall at anchor, have to always hang. uncomfortable, dangerous
\item Less heavy than dynamic lifeline, but similar 
\item Outside of rope system. Needs wide/thick nylon sling or aramid one, burning through hazard
\end{itemize} 

\subsubsection{Rope w/ clove hitch}
\begin{itemize}
\item More effort at anchor than lifeline, but can get efficient at it (practice both hands, both gate directions)
\item Very adjustable, even long distances. Clove hitch can be hard to untie
\item Dynamic, safe against slip/fall at anchor
\item Takes up no space on belay loop, can't get stuck behind cams/legs
\item Inside rope system, can't untie. Very safe, rope is stronger than slings
\end{itemize}

\subsubsection{Conclusion}
Nico (mountain guide) prefers clove hitch using the rope.\\
For myself I think using dynamic or static sling lifelines is a bad habit and adds unnecessary complexity.\\
You can try out all options and draw your own conclusion. \\
For clove hitch, tying with both strands instead of 1 is stronger and prevents rope asymmetry. Use a large screwgate (gravity loaded), ballocks can't be used one-handed.


